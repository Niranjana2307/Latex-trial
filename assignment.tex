\documentclass[10pt]{extarticle}
\usepackage[english]{babel}
\usepackage[T1]{fontenc}
\usepackage{lmodern,mathrsfs}
\usepackage{pythontex}
\usepackage{xparse}
\usepackage[inline,shortlabels]{enumitem}
\setlist{topsep=2pt,itemsep=2pt,parsep=0pt,partopsep=0pt}
\usepackage[dvipsnames]{xcolor}
\usepackage[utf8]{inputenc}
\usepackage[a4paper,top=0.5in,bottom=0.2in,left=0.5in,right=0.5in,footskip=0.3in,includefoot]{geometry}
\usepackage[most]{tcolorbox}
\tcbuselibrary{minted} % tcolorbox minted library, required to use the "minted" tcb listing engine (this library is not loaded by the option [most])
\usepackage{minted} % Allows input of raw code, such as Python code
\usepackage[colorlinks]{hyperref} % ALWAYS load this package LAST

% Custom tcolorbox style for Python code (not the code or the box it appears in, just the options for the box)
\tcbset{
    pythoncodebox/.style={
        enhanced jigsaw,breakable,
        colback=gray!10,colframe=gray!20!black,
        boxrule=1pt,top=2pt,bottom=2pt,left=2pt,right=2pt,
        sharp corners,before skip=10pt,after skip=10pt,
        attach boxed title to top left,
        boxed title style={empty,
            top=0pt,bottom=0pt,left=2pt,right=2pt,
            interior code={\fill[fill=tcbcolframe] (frame.south west)
                --([yshift=-4pt]frame.north west)
                to[out=90,in=180] ([xshift=4pt]frame.north west)
                --([xshift=-8pt]frame.north east)
                to[out=0,in=180] ([xshift=16pt]frame.south east)
                --cycle;
            }
        },
        title={#1}, % Argument of pythoncodebox specifies the title
        fonttitle=\sffamily\bfseries
    },
    pythoncodebox/.default={}, % Default is No title
    %%% Starred version has no frame %%%
    pythoncodebox*/.style={
        enhanced jigsaw,breakable,
        colback=gray!10,coltitle=gray!20!black,colbacktitle=tcbcolback,
        frame hidden,
        top=2pt,bottom=2pt,left=2pt,right=2pt,
        sharp corners,before skip=10pt,after skip=10pt,
        attach boxed title to top text left={yshift=-1mm},
        boxed title style={empty,
            top=0pt,bottom=0pt,left=2pt,right=2pt,
            interior code={\fill[fill=tcbcolback] (interior.south west)
                --([yshift=-4pt]interior.north west)
                to[out=90,in=180] ([xshift=4pt]interior.north west)
                --([xshift=-8pt]interior.north east)
                to[out=0,in=180] ([xshift=16pt]interior.south east)
                --cycle;
            }
        },
        title={#1}, % Argument of pythoncodebox specifies the title
        fonttitle=\sffamily\bfseries
    },
    pythoncodebox*/.default={}, % Default is No title
}

% Custom tcolorbox for Python code (not the code itself, just the box it appears in)
\newtcolorbox{pythonbox}[1][]{pythoncodebox=#1}
\newtcolorbox{pythonbox*}[1][]{pythoncodebox*=#1} % Starred version has no frame

% Custom minted environment for Python code, NOT using tcolorbox
\newminted{python}{autogobble,breaklines,mathescape}

% Custom tcblisting environment for Python code, using the "minted" tcb listing engine
% Adapted from https://tex.stackexchange.com/a/402096
\NewTCBListing{python}{ !O{} !D(){} !G{} }{
    listing engine=minted,
    listing only,
    pythoncodebox={#1}, % First argument specifies the title (if any)
    minted language=python,
    minted options/.expanded={
        autogobble,breaklines,mathescape,
        #2 % Second argument, delimited by (), denotes options for the minted environment
    },
    #3 % Third argument, delimited by {}, denotes options for the tcolorbox
}

%%% Starred version has no frame %%%
\NewTCBListing{python*}{ !O{} !D(){} !G{} }{
    listing engine=minted,
    listing only,
    pythoncodebox*={#1}, % First argument specifies the title (if any)
    minted language=python,
    minted options/.expanded={
        autogobble,breaklines,mathescape,
        #2 % Second argument, delimited by (), denotes options for the minted environment
    },
    #3 % Third argument, delimited by {}, denotes options for the tcolorbox
}

% verbbox environment, for showing verbatim text next to code output (for package documentation and user learning purposes)
\NewTCBListing{verbbox}{ !O{} }{
    listing engine=minted,
    minted language=latex,
    boxrule=1pt,sidebyside,skin=bicolor,
    colback=gray!10,colbacklower=white,valign=center,
    top=2pt,bottom=2pt,left=2pt,right=2pt,
    #1
} % Last argument allows more tcolorbox options to be added

\setlength{\parindent}{0.2in}
\setlength{\parskip}{0pt}
\setlength{\columnseprule}{0pt}

\makeatletter
% Redefining the title block
\renewcommand\maketitle{
    \null\vspace{4mm}
    \begin{center}
        {\Huge\sffamily\bfseries\selectfont\@title}\\
            \vspace{4mm}
        {\Large\sffamily\selectfont\@author}\\
            \vspace{4mm}
        {\large\sffamily\selectfont\@date}
    \end{center}
    \vspace{6mm}
}
% Adapted from https://tex.stackexchange.com/questions/483953/how-to-add-new-macros-like-author-without-editing-latex-ltx?noredirect=1&lq=1
\makeatother


\title{Introduction to Computer Science I - New SME Assignment Solution}
\author{Niranjana Srinivasa Ragavan}
\date{March 2025}

\begin{document}

\maketitle

In this document, I have written the explanations for the questions and solutions provided in the given assignment document. I have used PythonTex package to execute the python code written within the Latex.
My explanations are based on the assumption of prerequisite knowledge specified in the assignment.

\section*{KP1. Understanding Nested List Comprehensions}
\subsection*{G1.1 (E) Predicting the Output of a Nested List Comprehension}

\textbf{Question}

Consider the following code snippet

    \begin{python}
    nested_list = [[(i + j) ** 2 for j in range(1, 4)] for i in range(1, 4)]
    print(nested_list)
    \end{python}

What will be printed? \\

\noindent\textbf{Solution}

\begin{python}
Output: [[4, 9, 16], [9, 16, 25], [16, 25, 36]]
\end{python}  \\

\noindent\textbf{Explanation} \\

Recall that nested list comprehension consists of list comprehensions within a list comprehension. Expression inside \texttt{[]} is called list comprehension.
\begin{itemize}
    \item In this code, the outer list comprehension \colorbox{gray!20}{\texttt{for i in range(1, 4)}}
    which generates values $i = 1,2,3$.
    \item For each value of \texttt{i}, the inner list comprehension \colorbox{gray!20}{\texttt{ for j in range(1, 4)}}
    generates values $j = 1,2,3$.
    \item For each pair of \texttt{i} and \texttt{j}, the expression \colorbox{gray!20}{\texttt{(i + j) ** 2}} is computed.
    \item For each iteration of \texttt{i}, the computed result is stored as a sublist and the final result is the list of all these sublists.
    \item \textbf{Step by step evaluation}
    \begin{tcolorbox}[colback=gray!20, colframe=gray!50, sharp corners=southwest]
    \begin{pycode}
for i in range(1, 4):
    result = []
    print("\ni  | ", "j  | ", "(i+j)**2 | \n")
    for j in range(1, 4):
        print(i, "|", j, "|", (i + j) ** 2, "|")
        print("\n")
        result.append((i + j) ** 2)
    print("\nSublist for i = {0}: {1}\n".format(i, result))
    \end{pycode}
    \end{tcolorbox}
    \item \textbf{Scope of the variables:} 
    \begin{itemize}
        \item In python, in the nested \texttt{for} loop, the inner loop has the same scope as that of the enclosing scope. Thus, outer loop variables defined before the inner loop can be directly referenced by the inner loop.   
        \item While in nested list comprehension, each list comprehension creates its own local scope. 
        \item Still, the inner list comprehension can access the variables local to the outer comprehension. This is due to the property called \textbf{Closures}. The inner list comprehension can access the variables from the enclosing scope.
        \item In this code, the non-local variable \texttt{i} which is in the scope of outer comprehension,  can be accessed as a free variable by inner list comprehension. This means, \texttt{i} is not bound to the inner comprehension, but is stored in a closure. This repeats for each iteration of \texttt{i}.
        \item This can be evident from this line of byte-level code  \colorbox{gray!20}{\texttt{LOAD\_CLOSURE 1 (i)}}.
    \end{itemize}
    
\end{itemize} 
\bigskip  
\subsection*{G1.2 (M) Converting a \texttt{for} Loop to a Nested List Comprehension}

\textbf{Question}

Rewrite the following \texttt{for} loop as a nested list comprehension:  

    \begin{python}
    matrix = [[1, 2, 3], [4, 5, 6], [7, 8, 9]]
    new_matrix = []
    for row in matrix:
        new_row = []
        for num in row:
            new_row.append(num + 2)
        new_matrix.append(new_row)
    \end{python}

\noindent\textbf{Solution}

\begin{python}
new_matrix = [[num + 2 for num in row] for row in matrix]
\end{python}  \\

\noindent\textbf{Explanation}

\begin{itemize}
    \item \textbf{Explanation of nested \texttt{for} loops:} The outer \texttt{for} loop iterates through each row in the matrix.
    The inner loop iterates through each number in the row and creates a new row by appending $2$ to each number in the original row.
    The new rows are appended to create the new matrix. 
    \item \textbf{Creating list comprehension:} Replace the outer loop with \colorbox{gray!20}{\texttt{[..for row in matrix]}}. In the inner loop, for each \texttt{num} in row,  \colorbox{gray!20}{\texttt{num + $2$}} expression should be executed. Replace the inner loop  with \colorbox{gray!20}{\texttt{[num + $2$ for num in row]}}.
    \item Finally, all the modified rows are appended to create \texttt{new\_matrix}. The list comprehension implicitly builds the list and there is no need for  \colorbox{gray!20}{\texttt{append}} function.
   
   
\end{itemize}
\section*{KP2. Using Nested List Comprehensions with Strings}
\subsection*{G2.1 (E) Predicting Output for Uppercasing Nested Words}

\textbf{Question}

What will be printed when the following code is run?

    \begin{python}
    sentence = "I am happy"
    words = [[char.upper() for char in word] for word in sentence.split()]
    print(words)

    \end{python}

\noindent\textbf{Solution}

\begin{python}
Output: [['I'], ['A', 'M'], ['H', 'A', 'P', 'P', 'Y']]
\end{python}  \\

\noindent\textbf{Explanation}

\begin{itemize}
    \item Method \colorbox{gray!20}{\texttt{sentence.split()}} splits the sentence and creates a \textbf{list} of individual words.
    \item Methods \colorbox{gray!20}{\texttt{char.upper()}} turns lowercase character into an uppercase character.
    \item \textbf{Outer comprehension:} It iterates through each \texttt{word} in the list of individual words created by \colorbox{gray!20}{\texttt{.split()}} method.
    \item \textbf{Inner comprehension:} It iterates through each \texttt{char} in the \texttt{word} and converts the character into uppercase using \colorbox{gray!20}{\texttt{.upper()}} method. If the \texttt{char} is already in uppercase, it does nothing. For each word, the list of its uppercase characters is created.
    \item The final list is the collection of sublists of uppercase letters of individual words.
    \item \textbf{Step by step evaluation}
    \begin{tcolorbox}[colback=gray!20, colframe=gray!50, sharp corners=southwest]
    \begin{pycode}
sentence = "I am happy"
# Split sentence into words
words_list = sentence.split()
print("Words List:\n", words_list)  

# Process each word separately
result = []
for word in words_list:
  char_list=[]
  print("\nword | ","char | ","upper  | ".expandtabs(4))
  for char in word:
    print("\n{0} |   {1}  |   {2}   |  ".format(word, char, char.upper()).expandtabs(4))
    char_list.append(char.upper())  # Convert each char in word to uppercase
  print(f"\nSublist '{word}':", char_list)  
    \end{pycode}
    \end{tcolorbox} 
\end{itemize}



\section*{KP3. Using Nested List Comprehensions with Conditions}
\subsection*{G3.1 (E) Creating a Nested List of Filtered Words}

\textbf{Question}

What is the output of the following code snippet?

    \begin{python}
    words = [["cat", "elephant"], ["dog", "tiger"], ["fox", "giraffe"]]
    long_words = [[word for word in row if len(word) > 3] for row in words]
    print(long_words)
    \end{python}

\noindent\textbf{Solution}

\begin{python}
Output: [['elephant'], ['tiger'], ['giraffe']]
\end{python}  \\

\noindent\textbf{Explanation}

\begin{itemize}
    \item Method \colorbox{gray!20}{\texttt{len(word)}} finds the length of the word.
    \item \textbf{Outer comprehension:} It iterates through each sublist (\texttt{row}) in the list \texttt{words}.
    \item \textbf{Inner comprehension:} It iterates through each \texttt{word} in the \texttt{row} and selects the word if the length of the word is greater than $3$ \colorbox{gray!20}{\texttt{len(word)>3}}.
    The selected words form the sublist.
    \item The final list (\texttt{long\_words}) is the collection of sublists of the selected words.
    \item \textbf{Step by step evaluation}
    \begin{tcolorbox}[colback=gray!20, colframe=gray!50, sharp corners=southwest]
    \begin{pycode}
words = [["cat", "elephant"], ["dog", "tiger"], ["fox", "giraffe"]]
result = []
for row in words:
  long_words=[]
  print("\nrow |","word |")
  for word in row:
    if len(word)>3:
          print("\n{0} | {1} |".format(row, word))
          long_words.append(word) 
           # Convert each char in word to uppercase
  print(f"\nSublist:{long_words}")  

    \end{pycode}
    \end{tcolorbox}
    \item \textit{Note:} In this case, the elements are filtered based on a condition, thus, \texttt{if} condition is after the \texttt{for} loop.
    If each element is modified based on the conditions \texttt{(if-else)}, then the conditions should come before the \texttt{for} loop 
\end{itemize}
\section*{KP4. Writing Code Using Nested List Comprehensions}
\subsection*{G4.1 Writing Code for Creating Given Nested List}

\textbf{Question}

Write a Python program using a nested list comprehension to create a 3 x 4 grid filled with zeros and print it. \\

\noindent\textbf{Solution}

\begin{python}
grid = [[0 for _ in range(4)] for _ in range(3)]
print(grid)
\end{python}  \\

\noindent\textbf{Explanation}

\begin{itemize}
    \item Recall that Nested List Comprehension consists of list comprehensions within a list comprehension.
    \item Here, it has been asked to create $ 3*4$ grid, which means there should be $3$ rows and $4$ columns. 
    \item Remember that matrix is a list of rows, each row is a sublist which contains column values.
    \item In general, Outer loop is used to create empty sublists and Inner loop is used to add or modify the values in each sublist. 
    \item Outer comprehension should create $3$ empty rows  \colorbox{gray!20}{\texttt{for \_ in range(3)}}.
    \item Inner comprehension should add value $0$ to $4$ columns in each row.  \colorbox{gray!20}{\texttt{[0 for \_ in range(4)]}} 
    \item \textit{Note:} \texttt{\_} is just a regular variable, we can add any meaningful variable in that place. It is used to indicate that the loop variable is not relevant and necessary. In this case, the loop variable is not used inside the expression, thus \texttt{\_} is used.
    \item \textbf{Step by step evaluation}
    \begin{tcolorbox}[colback=gray!20, colframe=gray!50, sharp corners=southwest]
    \begin{pycode}
result = []
for i in range(3):
  list_=[]
  print("\nrow index | ","column index | ", "row |")
  for j in range(4):
      list_.append(0)
      print("\n{0}  |  {1}  |  {2}  |".format(i, j,list_))
  print(f"\nRow:{list_}")   

    \end{pycode}
    \end{tcolorbox}
\end{itemize}

\subsection*{G4.2 Writing Code for Creating Given Nested List from Strings}

\textbf{Question}

Write a nested list comprehension that extracts only vowels from each word in a sentence, storing them in nested lists. For \texttt{sentence = "Python Is Amazing"}, the output must be \texttt{[['o'], ['I'], ['A', 'a', 'i']]}

\noindent\textbf{Solution}

\begin{python}
sentence = "Python Is Amazing"
vowels = "aeiouAEIOU"
vowel_list = [[char for char in word if char in vowels] for word in sentence.split()]
print(vowel_list)
\end{python}  \\

\noindent\textbf{Explanation}

\begin{itemize}
    \item It has been asked to create a list of lists. Each sublist should contain only the vowels from each word of the given sentence.
    \item Remember that Method \colorbox{gray!20}{\texttt{sentence.split()}} splits the sentence and creates a \textbf{list} of individual words.
    \item Given sentence is \colorbox{gray!20}{\texttt{"Python is amazing"}}. 
    \item It is necessary to define the list (or any iterable) of vowels for the comparison. 
    \item Here, it is defined as string \colorbox{gray!20}{\texttt{vowels = "aeiouAEIOU"}}. Note that, in Python, string is an iterable. 
    \item Remember the outer loop is used to generate sublists, and inner loop is used to add or modify values in the sublist.
    \item Here, the number of sublists equal to the number of words. Thus, outer comprehension should split the sentence to generate words and an empty row for each word. \colorbox{gray!20}{\texttt{[[] for word in sentence.split()]]}} 
    \item The inner loop should select characters from each word, if the character is present in the \texttt{vowel} string. and fills the empty rows. 
    \item Remember if it is for \textbf{filtering} in the list comprehension, then the \texttt{if} condition should come after the \texttt{for} loop. \colorbox{gray!20}{\texttt{[expression for item in iterable if condition]}}  
    \item The inner comprehension should be \colorbox{gray!20}{\texttt{[char for char in word if char in vowels]}}.
    \item \textbf{Step by step evaluation}
    \begin{tcolorbox}[colback=gray!20, colframe=gray!50, sharp corners=southwest]
    \begin{pycode}
sentence = "Python Is Amazing"
vowels = "aeiouAEIOU"

# Split sentence into words
words_list = sentence.split()
print("\nWords List:", words_list)  

# Process each word separately

for word in words_list:
  char_list=[]
  print("\nword | ","char | ")
  for char in word:
    if char in vowels: 
      char_list.append(char)
      print("\n{0} | {1} |".format(word, char))
  print(f"\nSublist: {char_list}")  

    \end{pycode}
    \end{tcolorbox}
\end{itemize}
\subsection*{G4.3 Writing Code for Creating Given Nested List with Conditionals}

\textbf{Question}

Write a nested list comprehension that creates a 5×5 grid, but fills it with 1 if the sum of row and column indices is even, and 0 otherwise, and print it.

\noindent\textbf{Solution}

\begin{python}
grid = [[1 if (i + j) % 2 == 0 else 0 for j in range(5)] for i in range(5)]
print(grid)
\end{python}  \\

\noindent\textbf{Explanation}

\begin{itemize}
    \item Recall that Nested List Comprehension consists of list comprehensions within a list comprehension.
    \item Here, it has been asked to create $ 5*5$ grid, which means there should be $5$ rows and $5$ columns. 
    \item Remember that matrix is a list of rows, each row is a sublist which contains column values.
    \item In general, Outer loop is used to create empty sublists and Inner loop is used to add or modify the values in each sublist. 
    \item Outer comprehension should create $5$ empty rows  \colorbox{gray!20}{\texttt{[[]for i in range(5)}]}.
    \item Inner comprehension should populate the $5$ column values in each row. It should insert $1$ if the sum of row and column indices is even, else it 
    should insert $0$, in the columns in each row.
    \item The sum $i + j$ is even, if the remainder is $0$ when the $i + j$  is divided by $2$.
    \item The modulo operator \colorbox{gray!20}{\texttt{\%}} is used to get the remainder of division of two numbers. 
    \item Thus, the expression would be \colorbox{gray!20}{\texttt{(i+j)\%2}}.
    \item The condition to check whether the sum is even or not: \colorbox{gray!20}{\texttt{1 if (i+j)\%2==0 else 0}}.
    \item Remember if it is for \textbf{modifying} each value in the list comprehension, then the \texttt{if-else} condition should come before the \texttt{for} loop. \colorbox{gray!20}{\texttt{[expression if condition else expression for item in iterable]}}  
    \item Inner comprehension would be \colorbox{gray!20}{\texttt{[1 if (i+j)\%2==0 else 0 for j in range(5)]}}.
    \item \textbf{Step by step evaluation} 
    \begin{tcolorbox}[colback=gray!20, colframe=gray!50, sharp corners=southwest]
    \begin{pycode}
result = []
for i in range(5):
  list_=[]
  print("\nrow index | ","column index |", "expression |", "row")
  for j in range(5):
    if (i + j) % 2 == 0:
      list_.append(1)
    else:
      list_.append(0)
    print("\n{0}  |  {1}  |  {2}  |  {3}  |".format(i, j, (i+j)%2 , list_))
  print(f"\nRow:{list_}")    
 

    \end{pycode}
    \end{tcolorbox}
\end{itemize}

\end{document}

