\documentclass[10pt]{extarticle}
\usepackage[english]{babel}
\usepackage[T1]{fontenc}
\usepackage{lmodern,mathrsfs}
\usepackage{pythontex}
\usepackage{xparse}
\usepackage[inline,shortlabels]{enumitem}
\setlist{topsep=2pt,itemsep=2pt,parsep=0pt,partopsep=0pt}
\usepackage[dvipsnames]{xcolor}
\usepackage[utf8]{inputenc}
\usepackage[a4paper,top=0.5in,bottom=0.2in,left=0.5in,right=0.5in,footskip=0.3in,includefoot]{geometry}
\usepackage[most]{tcolorbox}
\tcbuselibrary{minted} % tcolorbox minted library, required to use the "minted" tcb listing engine (this library is not loaded by the option [most])
\usepackage{minted} % Allows input of raw code, such as Python code
\usepackage[colorlinks]{hyperref} % ALWAYS load this package LAST

% Custom tcolorbox style for Python code (not the code or the box it appears in, just the options for the box)
\tcbset{
    pythoncodebox/.style={
        enhanced jigsaw,breakable,
        colback=gray!10,colframe=gray!20!black,
        boxrule=1pt,top=2pt,bottom=2pt,left=2pt,right=2pt,
        sharp corners,before skip=10pt,after skip=10pt,
        attach boxed title to top left,
        boxed title style={empty,
            top=0pt,bottom=0pt,left=2pt,right=2pt,
            interior code={\fill[fill=tcbcolframe] (frame.south west)
                --([yshift=-4pt]frame.north west)
                to[out=90,in=180] ([xshift=4pt]frame.north west)
                --([xshift=-8pt]frame.north east)
                to[out=0,in=180] ([xshift=16pt]frame.south east)
                --cycle;
            }
        },
        title={#1}, % Argument of pythoncodebox specifies the title
        fonttitle=\sffamily\bfseries
    },
    pythoncodebox/.default={}, % Default is No title
    %%% Starred version has no frame %%%
    pythoncodebox*/.style={
        enhanced jigsaw,breakable,
        colback=gray!10,coltitle=gray!20!black,colbacktitle=tcbcolback,
        frame hidden,
        top=2pt,bottom=2pt,left=2pt,right=2pt,
        sharp corners,before skip=10pt,after skip=10pt,
        attach boxed title to top text left={yshift=-1mm},
        boxed title style={empty,
            top=0pt,bottom=0pt,left=2pt,right=2pt,
            interior code={\fill[fill=tcbcolback] (interior.south west)
                --([yshift=-4pt]interior.north west)
                to[out=90,in=180] ([xshift=4pt]interior.north west)
                --([xshift=-8pt]interior.north east)
                to[out=0,in=180] ([xshift=16pt]interior.south east)
                --cycle;
            }
        },
        title={#1}, % Argument of pythoncodebox specifies the title
        fonttitle=\sffamily\bfseries
    },
    pythoncodebox*/.default={}, % Default is No title
}

% Custom tcolorbox for Python code (not the code itself, just the box it appears in)
\newtcolorbox{pythonbox}[1][]{pythoncodebox=#1}
\newtcolorbox{pythonbox*}[1][]{pythoncodebox*=#1} % Starred version has no frame

% Custom minted environment for Python code, NOT using tcolorbox
\newminted{python}{autogobble,breaklines,mathescape}

% Custom tcblisting environment for Python code, using the "minted" tcb listing engine
% Adapted from https://tex.stackexchange.com/a/402096
\NewTCBListing{python}{ !O{} !D(){} !G{} }{
    listing engine=minted,
    listing only,
    pythoncodebox={#1}, % First argument specifies the title (if any)
    minted language=python,
    minted options/.expanded={
        autogobble,breaklines,mathescape,
        #2 % Second argument, delimited by (), denotes options for the minted environment
    },
    #3 % Third argument, delimited by {}, denotes options for the tcolorbox
}

%%% Starred version has no frame %%%
\NewTCBListing{python*}{ !O{} !D(){} !G{} }{
    listing engine=minted,
    listing only,
    pythoncodebox*={#1}, % First argument specifies the title (if any)
    minted language=python,
    minted options/.expanded={
        autogobble,breaklines,mathescape,
        #2 % Second argument, delimited by (), denotes options for the minted environment
    },
    #3 % Third argument, delimited by {}, denotes options for the tcolorbox
}

% verbbox environment, for showing verbatim text next to code output (for package documentation and user learning purposes)
\NewTCBListing{verbbox}{ !O{} }{
    listing engine=minted,
    minted language=latex,
    boxrule=1pt,sidebyside,skin=bicolor,
    colback=gray!10,colbacklower=white,valign=center,
    top=2pt,bottom=2pt,left=2pt,right=2pt,
    #1
} % Last argument allows more tcolorbox options to be added

\setlength{\parindent}{0.2in}
\setlength{\parskip}{0pt}
\setlength{\columnseprule}{0pt}

\makeatletter
% Redefining the title block
\renewcommand\maketitle{
    \null\vspace{4mm}
    \begin{center}
        {\Huge\sffamily\bfseries\selectfont\@title}\\
            \vspace{4mm}
        {\Large\sffamily\selectfont\@author}\\
            \vspace{4mm}
        {\large\sffamily\selectfont\@date}
    \end{center}
    \vspace{6mm}
}
% Adapted from https://tex.stackexchange.com/questions/483953/how-to-add-new-macros-like-author-without-editing-latex-ltx?noredirect=1&lq=1
\makeatother


\title{Introduction to Computer Science I - New SME Assignment Solution}
\author{Niranjana Srinivasa Ragavan}
\date{March 2025}

\begin{document}

\maketitle

In this document, I have written the explanations for the questions and solutions provided in the given assignment document. I have used PythonTex package to execute the python code written within the Latex.
My explanations are based on the assumption of prerequisite knowledge specified in the assignment.

\section*{KP1. Understanding Nested List Comprehensions}
\subsection*{G1.1 (E) Predicting the Output of a Nested List Comprehension}

\textbf{Question}

Consider the following code snippet

    \begin{python}
    nested_list = [[(i + j) ** 2 for j in range(1, 4)] for i in range(1, 4)]
    print(nested_list)
    \end{python}

What will be printed? \\

\noindent\textbf{Solution}

\begin{python}
Output: [[4, 9, 16], [9, 16, 25], [16, 25, 36]]
\end{python}  \\

\noindent\textbf{Explanation} \\

Recall that nested list comprehension consists of list comprehensions within a list comprehension. Expression inside \texttt{[ ]} is called list comprehension.
\begin{itemize}
    \item In this code, outer list comprehension \colorbox{gray!20}{\texttt{for i in range(1, 4)}}
    which generates values $i = 1,2,3$
    \item For each value of \texttt{i}, inner list comprehension \colorbox{gray!20}{\texttt{ for j in range(1, 4)}}
    generates values $j = 1,2,3$
    \item For each pair of \texttt{i} and \texttt{j}, the expression \colorbox{gray!20}{\texttt{(i + j) ** 2}} is computed.
    \item For each iteration of \texttt{i}, the computed result is stored as a sublist and the final result is the list of all these sublists
    \item \textbf{Step by step evaluation}
    \begin{tcolorbox}[colback=gray!20, colframe=gray!50, sharp corners=southwest]
    \begin{pycode}
for i in range(1, 4):
    result = []
    print("\ni", "j", "(i+j)**2\n")
    for j in range(1, 4):
        print(i, j, (i + j) ** 2)
        print("\n")
        result.append((i + j) ** 2)
    print("\nSublist for i = {0}: {1}\n".format(i, result))
    \end{pycode}
    \end{tcolorbox}
\end{itemize}


\subsection*{G1.2 (M) Converting a \texttt{for} Loop to a Nested List Comprehension}

\textbf{Question}

Rewrite the following \texttt{for} loop as a nested list comprehension:  

    \begin{python}
    matrix = [[1, 2, 3], [4, 5, 6], [7, 8, 9]]
    new_matrix = []
    for row in matrix:
        new_row = []
        for num in row:
            new_row.append(num + 2)
        new_matrix.append(new_row)
    \end{python}

\noindent\textbf{Solution}

\begin{python}
new_matrix = [[num + 2 for num in row] for row in matrix]
\end{python}  \\

\noindent\textbf{Explanation}

\section*{KP2. Using Nested List Comprehensions with Strings}
\subsection*{G2.1 (E) Predicting Output for Uppercasing Nested Words}

\textbf{Question}

What will be printed when the following code is run?

    \begin{python}
    sentence = "I am happy"
    words = [[char.upper() for char in word] for word in sentence.split()]
    print(words)

    \end{python}

\noindent\textbf{Solution}

\begin{python}
Output: [['I'], ['A', 'M'], ['H', 'A', 'P', 'P', 'Y']]
\end{python}  \\

\noindent\textbf{Explanation}

\section*{KP3. Using Nested List Comprehensions with Conditions}
\subsection*{G3.1 (E) Creating a Nested List of Filtered Words}

\textbf{Question}

What is the output of the following code snippet?

    \begin{python}
    words = [["cat", "elephant"], ["dog", "tiger"], ["fox", "giraffe"]]
    long_words = [[word for word in row if len(word) > 3] for row in words]
    print(long_words)
    \end{python}

\noindent\textbf{Solution}

\begin{python}
Output: [['elephant'], ['tiger'], ['giraffe']]
\end{python}  \\

\noindent\textbf{Explanation}

\section*{KP4. Writing Code Using Nested List Comprehensions}
\subsection*{G4.1 Writing Code for Creating Given Nested List}

\textbf{Question}

Write a Python program using a nested list comprehension to create a 3 x 4 grid filled with zeros and print it. \\

\noindent\textbf{Solution}

\begin{python}
grid = [[0 for _ in range(4)] for _ in range(3)]
print(grid)
\end{python}  \\

\noindent\textbf{Explanation}

\subsection*{G4.2 Writing Code for Creating Given Nested List from Strings}

\textbf{Question}

Write a nested list comprehension that extracts only vowels from each word in a sentence, storing them in nested lists. For \texttt{sentence = "Python Is Amazing"}, the output must be \texttt{[['o'], ['I'], ['A', 'a', 'i']]}

\noindent\textbf{Solution}

\begin{python}
sentence = "Python Is Amazing"
vowels = "aeiouAEIOU"
vowel_list = [[char for char in word if char in vowels] for word in sentence.split()]
print(vowel_list)
\end{python}  \\

\noindent\textbf{Explanation}

\subsection*{G4.3 Writing Code for Creating Given Nested List with Conditionals}

\textbf{Question}

Write a nested list comprehension that creates a 5×5 grid, but fills it with 1 if the sum of row and column indices is even, and 0 otherwise, and print it.

\noindent\textbf{Solution}

\begin{python}
grid = [[1 if (i + j) % 2 == 0 else 0 for j in range(5)] for i in range(5)]
print(grid)
\end{python}  \\

\noindent\textbf{Explanation}

\end{document}

